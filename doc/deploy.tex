

\section{Distribución}

En el directorio de la práctica se pueden ver diversos ficheros y directorios los cuales se explican de esta forma:

\begin{itemize}
\item *.cpp y *.h: Ficheros fuente de la aplicación.
\item bin: Se almacenan los binarios de la aplicación.
\item doc: Esta documentación.
\item resources: Se almacenan los datos fuente como las matrices origen y el número de procesadores.
\item config : Contiene los distintos ficheros de configuración para los nodos.
\item *.sh : Diferentes scripts para desplegar, ejecutar y parar la aplicación distribuida.
\item CannonApp.xml : Aplicación distribuida.
\end{itemize}

\section{Despliegue}

Para realizar el despliegue de la aplicación se han definido los siguientes nodos:
\begin{itemize}

\item Nodo1 : contiene el \emph{IceGrid Registry} además de un servidor \emph{IcePatch}. También se le incluyen siete procesos \emph{MatrixProcessor} y el \emph{MatrixCollector}.
\item Nodo2 : contiene nueve \emph{MatrixProcessor}.
\item Nodo3 : contiene nueve \emph{MatrixProcessor}.
\end{itemize}



\section{Compilado y Ejecución}

Para compilar simplemente basta realizar en el directorio de la práctica ``make''.
Por supuesto hay que tener instalado los siguientes paquetes:
\begin{itemize}
\item ZeroC Ice 3.4 ó superior.
\item GNU Make
\item GNU Compiler Collection
\end{itemize}

Esto generará en la carpeta bin los siguientes binarios:
\begin{itemize}
\item \textbf{MatrixClient:} cliente de la aplicación.
\item \textbf{MatrixCollector:} ejecutable del proceso \emph{Collector}.
\item \textbf{MatrixProcessor:} ejecutable del proceso \emph{Processor}.
\item \textbf{generate:} programa de generación de matrices. Se ejecuta desde el directorio principal de la práctica y toma dos argumentos. Estos son dos números enteros indicando el orden de las dos matrices resultantes, que se situarán en la carpeta resources con los nombres de ``matrizA.m'' y ``matrizB.m''. Su ejecución sería: ``./bin/generate <num> <num>''.
\end{itemize}

\paragraph{A tener en cuenta.}


Antes de todo, el directorio de esta práctica se debe de copiar en el directorio /tmp del sistema. Esto es debido a que el despliegue anterior se realiza mediante \emph{IcePatch} y este conoce el directorio de los binarios, el cual es /tmp/distribuidos/bin.

\paragraph{}
Para ejecutar la aplicación , que se debe de realizar desde el directorio principal de la práctica, simplemente lanzamos los scripts.
Realizando ``start.sh'' se despliega y empiezan a ejecutar los distintos nodos y servicios. \\
Tras esto podemos realizar la ejecución por parte del cliente, la cual se ha facilitado creando una regla en el Makefile. Por lo tanto su ejecución sería `` make ejecuteClient''. Una vez realizada la ejecución obtendremos en el mismo directorio los ficheros ``cannon\_result.m'' y ``normal\_result.m''. \\
Para parar la ejecución del sistema distribuido simplemente habría que ejecutar el script ``stop.sh''. \\
Se incluye también un script ``clean.sh'' para limpiar todos los ficheros de log, y las bases de datos usadas por los nodos, por lo que antes de cada ejecución es recomendable su ejecución.
